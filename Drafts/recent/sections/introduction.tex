% Import the subfiles package, which allows the document to be
% split into smaller subfiles for easier management
\documentclass[../main.tex]{subfiles}

% Begin the main document
\begin{document}

% Create a section titled "Introduction" and underline it using
% the soul package
\section{Introduction}

\textbf{Medretur} wants to develop a medicine collection
system that automates the process of disposing of medicines.
The task of this project is to research and find methods for
this system to detect and deter intrusion attempts. This
chapter presents background information on the research
project, including its objectives and limitations. Information
is also provided on the company that commissioned the study.

% Create a subsection titled "Background of the project" and underline
% it using the soul package
\subsection{Background}

The proper disposal of unused and expired medication is a
critical public health issue\cite{Throckmorton_2021}. Improper
disposal of such medicines can lead to contamination of the
environment and potential harm to human health\cite{Boxall_2004,
/content/publication/3854026c-en}.
Collection programs for unused medicines have been established to
address this issue\cite{Apotek_1_2023, Commissioner_2021, Helsenorge_2019},
but these programs often face challenges in terms of security\cite{Lutro_2005}.
This is because these programs deal with
sensitive materials that may be valuable to others\cite{Nations_2023}.

% Create a subsection titled "Purpose of the project" and underline it
% using the soul package
\subsection{Objectives}
The purpose of this task is to find a solution to prevent
theft and break-ins in systems for collecting medicines.

The aim of the proposed solution is to allow such systems to
operate without the risk of medications being lost or stolen.

The task aims to evaluate different detection and notification
methods and create a scaled-down prototype,
and then conclude whether or not this solution works.

\subsection{Limitations}
Parts of a control system that use the same components as the
existing control system in the \gls{medicolbox} will be constructed. 
%TODO: add and cite to a figure that shows the connection of components in use

The full-scale control system will have the same \gls{ccu},
\gls{hmi}, and \gls{sensor-card}, but also \gls{rio} for the \glspl{actuator}.

The prototype that will be assembled is a \gls{sbc}
Raspberry Pi 4B 8GB\cite{pi} as \gls{ccu}, with a 7-inch touchscreen\cite{pi-screen} as \gls{hmi}
and a Raspberry Pi Sense Hat v1.0.0\cite{pi-sense-hat} as a \gls{sensor-card}.
The purpose of constructing the prototype is to use the \gls{imu} in
the \gls{sensor-card}.

% Create a subsection titled "Target audience" and underline it using
% the soul package
\subsection{Project Stakeholders}

The company \textbf{Medretur} is the main target audience.
Medretur has a policy to have a long-standing commitment to
providing safe and effective waste management solutions to their
customers\cite{medretur-vision}.
However, the waste management industry is complex and
challenging, and delivering a solution that reliably provides
this is not easy. The proposed project seeks to help Medretur
navigate this complex landscape by developing a new system that
will improve the safety and security of their waste management
operations.
On a long term basis, there will be developed a system that will
help to streamline the waste collection process, reduce the risk
of human error, and ensure that all waste is managed in accordance
with the current safety and regulatory standards.

By implementing this new system, Medretur will be able to reduce
the risk of accidents, errors, and other safety incidents, which
will help to protect their employees, customers, and the
environment.
Overall, the solution proposed by Medretur has the potential to provide
significant benefits to Medretur and their customers.

% End the document
\end{document}
