\documentclass[../main.tex]{subfiles}

\begin{document}
    % Add an abstract to the document,
    % with a citation using the cite package
    \begin{abstract}
        \subsection*{EN - english}
        This study focuses on the development of a solution to prevent theft and break-ins in medicine collection systems, specifically targeting the \gls{medicolbox}. The proper disposal of unused and expired medications is crucial for public health, as improper disposal can lead to environmental contamination and potential harm to individuals. The objective of this research is to find methods to detect and deter intrusion attempts in these systems, aiming to minimize the risk of medication loss or theft. A scaled-down prototype utilizing an \gls{imu} is constructed to evaluate different detection and notification methods. The study examines accelerometer data collected during simulated intrusion events, including striking, dropping, sawing, and shaking of the \gls{medicolbox}. The results demonstrate the effectiveness of the \gls{imu} in detecting and capturing accelerative patterns associated with these intrusion scenarios. However, it is noted that the \gls{imu}'s sensitivity may lead to false alarms, affecting the system's accuracy. The study also discusses the potential of combining the accelerometer data with gyroscope data to understand the relationship between acceleration and orientation during intrusion events. The findings supports the feasibility of utilizing the \gls{imu}, but not as a standalone solution, even thoigh it as a standalone solution will enhance the security and protection of the \gls{medicolbox}. Further research directions include integrating other sensor types, and exploring the applicability of this intrusion detection approach to other systems beyond medicine collection boxes. The outcomes of this study contribute to improving the overall security and integrity of portable storage systems and preventing unauthorized access.

        \subsection*{NO - Norsk}
        Denne studien fokuserer på utviklingen av en løsning for å forebygge tyveri og innbrudd i systemer for innsamling av medisiner, spesifikt rettet mot \gls{medicolbox}. Forsvarlig håndtering av ubrukte og utgåtte medisiner er avgjørende for folkehelsen, ettersom feilaktig håndtering kan føre til miljøforurensning og potensiell skade på individer. Målet med denne forskningen er å finne metoder for å detektere og avskrekke innbruddsforsøk i disse systemene, med mål om å minimere risikoen for tap eller tyveri av medisiner. En nedskalert prototype som bruker en \gls{imu}, er konstruert for å evaluere forskjellige deteksjons- og varslingsteknikker. Studien undersøker akselerasjonsdata samlet inn under simulerte innbruddshendelser, inkludert slag, fall, sageforsøk og risting av \gls{medicolbox}. Resultatene demonstrerer effektiviteten til \gls{imu}-en i å detektere og fange opp akselerasjonsmønstre assosiert med disse innbruddsscenariene. Det bemerkes imidlertid at \gls{imu}-ens følsomhet kan føre til falske alarmer, noe som påvirker systemets nøyaktighet. Studien diskuterer også potensialet ved å kombinere akselerometerdata med gyroskopdata for å forstå forholdet mellom akselerasjon og orientering under innbruddshendelser. Funnene støtter gjennomførbarheten av å bruke \gls{imu}-en, men ikke som en enkeltstående løsning, selv om den som en enkeltstående løsning vil forbedre sikkerheten og beskyttelsen av \gls{medicolbox}. Videre forskningsretninger inkluderer integrasjon av andre sensortyper og utforskning av anvendeligheten av denne tilnærmingen til innbruddsdeteksjon for andre systemer utover bokser for innsamling av medisin. Resultatene av denne studien bidrar til å forbedre den samlede sikkerheten og integriteten til bærbare lagringssystemer og forhindre uautorisert tilgang.

    \end{abstract}
\end{document}
