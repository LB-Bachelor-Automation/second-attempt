\documentclass[../main.tex]{subfiles}

\begin{document}

\clearpage
\section{Conclusion}
The analysis of the accelerometer data during the simulated intrusion attempts on the MediColbox revealed several key findings. Firstly, the IMU demonstrated remarkable sensitivity to changes in acceleration, successfully capturing both high-impact events and subtle vibrations. The accelerometer readings accurately detected the accelerative patterns associated with striking, dropping, sawing, and shaking the MediColbox, indicating the effectiveness of the IMU in detecting and capturing different intrusion scenarios.

However, it was observed that the IMU's high sensitivity also led to the detection of movements from nearby individuals, which may result in false alarms. To address this limitation, further research is recommended to explore the implementation of pattern recognition software, such as machine learning algorithms, to filter out false alarms and improve the accuracy of intrusion detection.

Additionally, the analysis highlighted the potential of combining the accelerometer data with other sensor types, such as gyroscope data, to gain a comprehensive understanding of both acceleration and orientation during intrusion events. This integration could enhance the overall security and integrity of the MediColbox system by providing a more holistic approach to intrusion detection.

The feasibility of utilizing the IMU as a standalone solution for enhancing the security and protection of the MediColbox is strongly supported by the study's findings. However, it is important to address the challenges associated with false alarms and explore the potential of integrating multiple sensor types to further improve the intrusion detection capabilities.

The implications of this study's findings extend beyond the MediColbox system. The successful application of the IMU in detecting intrusion attempts suggests its potential use in other portable storage systems or similar applications that require intrusion prevention. Future research should investigate the adaptability of this intrusion detection approach to different contexts and explore its broader implications.

In conclusion, the analysis of the accelerometer data during the simulated intrusion attempts provides valuable insights into the effectiveness of the IMU in detecting unauthorized access. The study lays the foundation for further research in the field of intrusion detection, including the development of more advanced algorithms and the exploration of potential applications beyond the MediColbox system. By improving the security and integrity of portable storage systems, these findings contribute to the protection of sensitive information and resources.
\end{document}