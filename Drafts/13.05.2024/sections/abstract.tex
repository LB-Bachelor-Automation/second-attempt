\documentclass[../main.tex]{subfiles}

\begin{document}
    % Add an abstract to the document,
    % with a citation using the cite package
    \begin{abstract}
        \subsection*{EN - english}
        This study focuses on the development of a solution to prevent theft and break-ins in medicine collection systems, specifically targeting the \gls{medicolbox}. The proper disposal of unused and expired medications is crucial for public health, as improper disposal can lead to environmental contamination and potential harm to individuals. The objective of this research is to find methods to detect and deter intrusion attempts in these systems, aiming to minimize the risk of medication loss or theft. A scaled-down prototype utilizing an \gls{imu} is constructed to evaluate different detection and notification methods. The study examines accelerometer data collected during simulated intrusion events, including striking, dropping, sawing, and shaking of the \gls{medicolbox}. The results demonstrate the effectiveness of the \gls{imu} in detecting and capturing accelerative patterns associated with these intrusion scenarios. However, it is noted that the \gls{imu}'s sensitivity may lead to false alarms, necessitating the exploration of pattern recognition software and machine learning algorithms to enhance the system's accuracy. The study also discusses the potential of combining the accelerometer data with gyroscope data to understand the relationship between acceleration and orientation during intrusion events. The findings strongly support the feasibility of utilizing the \gls{imu} as a standalone solution for enhancing the security and protection of the \gls{medicolbox}. Further research directions include integrating other sensor types, such as the gyroscope, and exploring the applicability of this intrusion detection approach to other systems beyond medicine collection boxes. The outcomes of this study contribute to improving the overall security and integrity of portable storage systems and preventing unauthorized access.

        \subsection*{NO - Norsk}
        Denne studien fokuserer på utviklingen av en løsning for å hindre tyveri og innbrudd i medisinske innsamlingssystemer, spesielt rettet mot \gls{medicolbox}. Riktig håndtering av ubrukte og utløpte medisiner er avgjørende for folkehelsen, da feilaktig håndtering kan føre til miljøforurensning og potensiell skade på enkeltpersoner. Målet med denne forskningen er å finne metoder for å oppdage og avverge innbruddsforsøk i disse systemene, med sikte på å minimere risikoen for tap eller tyveri av medisiner. En nedskalert prototype som bruker en \gls{imu} er konstruert for å evaluere ulike metoder for deteksjon og varsling. Studien undersøker akselerometerdata som er samlet inn under simulerte innbruddshendelser, inkludert slag, dropp, såing og risting av \gls{medicolbox}. Resultatene viser effektiviteten til \gls{imu} i å oppdage og fange opp akselerasjonsmønstre knyttet til disse innbruddsscenarioene. Imidlertid bemerkes det at \gls{imu}s følsomhet kan føre til falske alarmer, og det er derfor behov for utforsking av mønstergjenkjenning og maskinlæringsalgoritmer for å forbedre nøyaktigheten til systemet. Studien diskuterer også potensialet for å kombinere akselerometerdata med gyroskopdata for å forstå sammenhengen mellom akselerasjon og orientering under innbruddshendelser. Funnene støtter sterkt muligheten for å bruke \gls{imu} som en frittstående løsning for å forbedre sikkerheten og beskyttelsen av \gls{medicolbox}. Videre forskningsretninger inkluderer integrering av andre sensortyper, som gyroskopet, og utforsking av anvendeligheten av denne innbruddsdeteksjonsmetoden på andre systemer utenfor medisinske innsamlingsskap. Resultatene fra denne studien bidrar til å forbedre den generelle sikkerheten og integriteten til bærbare lagringssystemer og forhindre uautorisert tilgang.
    \end{abstract}
\end{document}
